\chapter{Introduction}\label{chapter:introduction}

This is the first chapter of the \gls{thesis}.~\cite{Aaboud:2016mmw,Bruning:782076}

\begin{figure}[htpb]
 \centering
 \includegraphics{introduction/example.pdf}
 \caption{This is a placeholder figure to act as an example.
  Here we cite a new reference in the caption to demonstrate that given the package configuration our order of references will not be distributed by the table of contents.~\cite{Higgs:1964ia}}\label{fig:test_figure}
\end{figure}

As can be seen in \Cref{fig:subfigure_example}, the subfigures are independent of each other such that \Cref{fig:subfigure_1} and \Cref{fig:subfigure_2} can be accessed separately.

\begin{figure}[htbp]
 \centering
 \begin{subfigure}[t]{0.5\textwidth}
  \centering
  \includegraphics[width=0.3\textwidth]{introduction/example.pdf}
  \caption{This is the first figure of two, in this example, and it its own independent subfigure.}
  \label{fig:subfigure_1}
 \end{subfigure}%
 ~
 \begin{subfigure}[t]{0.5\textwidth}
  \centering
  \includegraphics[width=0.3\textwidth]{introduction/example.pdf}
  \caption{As the \texttt{t} alignment option was chosen for the subfigures, they are still properly aligned vertically even though this caption is longer.}
  \label{fig:subfigure_2}
 \end{subfigure}
 \caption{An example of a figure that consists of two subfigures.}
 \label{fig:subfigure_example}
\end{figure}

As an example of an equation formatted in ``\href{https://www.overleaf.com/learn/latex/Display_style_in_math_mode}{display style}'' the equation for the fiducial cross section from~\cite{Aaboud:2016mmw} is reproduced as \Cref{eq:fiducial_cross_section}:

\begin{equation}
 \sigma_{\mathrm{inel}}^{\mathrm{fid}} \left(\zeta > 10^{-6}\right) = \frac{N - N_{\mathrm{BG}}}{\epsilon_{\mathrm{trig}} \times \mathcal{L}} \times \frac{1 - f_{\zeta < 10^{-6}}}{\epsilon_{\mathrm{sel}}}
 \label{eq:fiducial_cross_section}
\end{equation}

\section{Dealing with Widows and Orphans}

To reduce the difficulty of dealing with widowed text (the last line of a paragraph at the start of a page) and orphaned text (the first line of paragraph at the end of a page) the \href{https://ctan.org/pkg/nowidow?lang=en}{\texttt{nowidow}} package is used.
However, that doesn't solve the issue of orphaned section titles.
The user must manually do this, but the following \href{https://texfaq.org/FAQ-widows}{simple advice from \TeX{} FAQ} is recommended:

\begin{quote}
 Once you've exhausted the automatic measures, and have a final draft you want to ``polish'', you should proceed to manual measures.
 To get rid of an orphan is simple: precede the paragraph with \texttt{\textbackslash clearpage} and the paragraph can’t start in the wrong place.
\end{quote}
