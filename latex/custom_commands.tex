% Custom Commands

% Mathematics related
\providecommand{\abs}[1]{\ensuremath\left|{#1}\right|}% The absolute value of the enclosed

%======================================================================
% BEGIN SMU Custom Commands
%======================================================================
% The many of the following commands were originally in the smu_thesis.sty
% file. Commands and comments were moved here to make style file more about
% document style. While custom environments can technically be considered
% style, the student created environments are included here to simplify
% the style file. Often these commands were created by students over the
% years and placed in the style file.  The commands have not been
% verified and can be obsolete LaTeX commands. The Commands can be used
% or ignored.


% BEGIN TCM Commands to set formatting for Paragraph section to have line
%           feed after heading & Numbered. This custom command makes
%           \paragraph, in essence, a form of \subsubsubsection.
%           Uncomment to use.
%
%  \makeatletter
%  \renewcommand\paragraph{\@startsection{paragraph}{4}{\z@}%
%    {-3.25ex\@plus -1ex \@minus -.2ex}%
%    {1.5ex \@plus .2ex}%
%    {\normalfont\normalsize\bfseries}}
%  \makeatother
%
% END of Commands to set formatting for Paragraph section to have line
%       feed after heading TCM

% eliminate compile warnings for noindent hfill hbox in PDF bookmarks
\pdfstringdefDisableCommands{
 \def\hbox{ }% hyperref (temporarily) changes \hbox to a space in this context.
 \def\hfill{ }% hyperref (temporarily) changes \hfill to a space in this context.
 \def\noindent{ }% hyperref (temporarily) changes \noindent to a space in this context.
 \def\hskip{ }% hyperref (temporarily) changes \hskip to a space in this context.
}

% Misc. Commands defined by students
\newcommand{\qed}{\mbox{$\ \Box$}}
\newcommand{\as}{\mbox{$/\hspace{-2mm}>$}}

\newcommand{\chap}[1]{\chapter{\protect\textbf{#1}}}

% theorem definitions
\newtheorem{ex}{Example}[chapter]
\newtheorem{theor}{Theorem}[chapter]
\newtheorem{mtd}{Method}[chapter]
\newtheorem{mydef}{Definition}[chapter]
\newtheorem{mylem}[theor]{Lemma}

% environment definitions
\newenvironment{example}{\begin{ex} \rm}{\hfill \qed \end{ex}}
\newenvironment{theorem}{\begin{theor}}{ \end{theor}}
\newenvironment{method}{\begin{mtd} \rm}{\hfill \qed \end{mtd}}
\newenvironment{definition}{\begin{mydef} \rm}{\hfill \qed \end{mydef}}
\newenvironment{lemma}{\begin{mylem}}{ \end{mylem}}
\newenvironment{proof}[0]{\vspace{0.3cm}\noindent {\bf Proof:} \rm }{\hfill \lqed}
\newenvironment{proof-of}[1]{\noindent {\bf Proof of #1:}}{\hfill \lqed}
%----------------------------------------------------------------------
% END Misc. Commands defined by students
%----------------------------------------------------------------------
%======================================================================
% END SMU Custom Commands
%======================================================================
\providecommand{\printercalibration}{%
 \begin{tikzpicture}[remember picture,overlay]
  % 1 inch printer calibration
  \draw[line width=4pt,red,arrows=->] (current page.north west) ++(1in,0)
  -- node[midway,anchor=east,red] {top = 1 in} ++(0,-1in)
  -- node[midway,anchor=north,red] {side = 1 in}
  ++(-1in,0);
  \draw[line width=4pt,red,arrows=->] (current page.north east) ++(-1in,0)
  -- ++(0,-1in)
  -- ++(1in,0);
  \draw[line width=4pt,red,arrows=->] (current page.south east) ++(-1in,0)
  -- ++(0,1in)
  -- ++(1in,0);
  \draw[line width=4pt,red,arrows=->] (current page.south west) ++(1in,0)
  -- ++(0,1in)
  -- ++(-1in,0);
  % 2 inch printer calibration
  \draw[line width=4pt,blue,arrows=->] (current page.north)
  -- node[pos=0.7,anchor=west,blue] {2 in}
  ++(0,-2in);
  \draw[line width=4pt,blue,arrows=->] (current page.east)
  -- node[pos=0.7,anchor=south,blue] {2 in}
  ++(-2in,0);
  \draw[line width=4pt,blue,arrows=->] (current page.south)
  -- node[pos=0.7,anchor=west,blue] {2 in}
  ++(0,2in);
  \draw[line width=4pt,blue,arrows=->] (current page.west)
  -- node[pos=0.7,anchor=south,blue] {2 in}
  ++(2in,0);
  % 2 inch box
  \filldraw (current page.center) circle (1pt);
  \node [draw, thick, blue, shape=rectangle, minimum width=2in, minimum height=2in, anchor=center] at (current page.center) {};
  \draw (current page.center) ++(0,1in) node [blue, above] {2 inches};
  % 1 inch box
  \node [draw, thick, red, shape=rectangle, minimum width=1in, minimum height=1in, anchor=center] at (current page.center) {};
  \draw (current page.center) ++(0,0.5in) node [red, above] {1 inch};
 \end{tikzpicture}
}
